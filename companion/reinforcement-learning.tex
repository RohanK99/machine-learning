\documentclass[12pt]{article}

\usepackage[utf8]{inputenc}
\usepackage[margin = 1in]{geometry}
\usepackage[english]{babel}
\usepackage{amsthm}
\usepackage{amssymb}
\usepackage{amsmath}
\usepackage{changepage}
\usepackage[makeindex]{imakeidx}
\usepackage{titlesec}
\usepackage{textcomp}
\usepackage{graphicx}
\usepackage{gensymb}
\usepackage{xcolor, soul}
\usepackage{hyperref}
\usepackage{pgfplots}
\usepackage{parskip}

\graphicspath{ {./images/} }

\definecolor{linkColour}{RGB}{140, 25, 57}
\definecolor{urlColour}{RGB}{137, 62, 27}

\setcounter{section}{-1}
\setcounter{secnumdepth}{4}

\titleformat{\paragraph}
{\normalfont\normalsize\bfseries}{\theparagraph}{1em}{}
\titlespacing*{\paragraph}
{0pt}{3.25ex plus 1ex minus .2ex}{1.5ex plus .2ex}

\hypersetup{colorlinks, citecolor=purple, filecolor=purple, linkcolor=linkColour, urlcolor=urlColour}

\makeindex

\newenvironment{fact*}[2][]
    {
    \begin{adjustwidth}{1em}{0em}
    \noindent
    \textbf{#2} \hfill #1
    
    \vspace{0.1in}
    \noindent
    \ignorespaces
    } {
    \end{adjustwidth}
    }

\newenvironment{fact}[2][]
    {
    \index{#2}
    \hypertarget{#2}{\vspace{0.2in}}
    \begin{adjustwidth}{1em}{0em}
    \noindent
    \textbf{#2} \hfill #1
    
    \vspace{0.1in}
    \noindent
    \ignorespaces
    } {
    \end{adjustwidth}
    }

\title{Companion to Reinforcement Learning}
\author{Rohan Kumar}
\date{}

\begin{document}

\maketitle
\newpage
\tableofcontents
\newpage

\section{Notation}
    \subsection{Data}
    \begin{flalign*}
        \boldsymbol{x} &= \begin{pmatrix} x_1 \\ x_2 \\ ... \\ x_M \end{pmatrix} \text{: data point corresponding to a
        column vector of $M$ features} & \\
        \overline{\boldsymbol{x}} &= \begin{pmatrix} 1 \\ x_1 \\ x_2 \\ ... \\ x_M \end{pmatrix} \text{: concatenation
        of 1 with the vector} \boldsymbol{x} & \\
        \boldsymbol{X} &= \begin{pmatrix} x_{1,1} & ... & x_{1,N} \\ ... & ... & ... \\ x_{M,1} & ... & x_{M,N}
        \end{pmatrix} \text{: dataset consisting of $N$ data points and $M$ features} & \\
        \overline{\boldsymbol{X}} &= \begin{pmatrix} 1 & ... & 1 \\ x_{1,1} & ... & x_{1,N} \\ ... & ... & ... \\
        x_{M,1} & ... & x_{M,N} \end{pmatrix} \text{: concatenation of a vector of 1's with the matrix } \boldsymbol{X}
        & \\
        y &= \text{: output target (regression) or label (classification)} & \\
        \boldsymbol{y} &= \begin{pmatrix} y_1 \\ y_2 \\ ... \\ y_N \end{pmatrix} \text{: vector of outputs for a dataset
        of $N$ points} & \\
        \boldsymbol{x}_* &= \text{: test input / unknown input } & \\
        \boldsymbol{y}_* &= \text{: predicted output} & \\
        N &= \text{: Number of data points in the dataset} & \\
        M &= \text{: Number of a features in a data point} & \\
        \boldsymbol{w} &= \begin{pmatrix} w_1 \\
            w_2 \\
            ... \\
            w_M \\
        \end{pmatrix} & \\
        \boldsymbol{w}^T &= (w_1, w_2, ..., w_M) \text{ or } (w_0, w_1, w_2, ..., w_M) \text{ $w_0$ multiplies the first
        entry of $\overline{\boldsymbol{x}}$ (bias)} & \\
    \end{flalign*}
    Note: bold symbols represents a vector

\printindex

\end{document}