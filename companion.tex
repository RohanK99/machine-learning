\documentclass[12pt]{article}

\usepackage[utf8]{inputenc}
\usepackage[margin = 1in]{geometry}
\usepackage[english]{babel}
\usepackage{amsthm}
\usepackage{amssymb}
\usepackage{amsmath}
\usepackage{changepage}
\usepackage[makeindex]{imakeidx}
\usepackage{textcomp}
\usepackage{gensymb}
\usepackage{xcolor, soul}
\usepackage{hyperref}

\definecolor{linkColour}{RGB}{140, 25, 57}
\definecolor{urlColour}{RGB}{137, 62, 27}

\hypersetup{
    colorlinks,
    citecolor=purple,
    filecolor=purple,
    linkcolor=linkColour,
    urlcolor=urlColour
}

\makeindex

\newcommand{\homl}{\hyperlink{homl}{Hands-On Machine Learning}}

\newenvironment{fact*}[2][]
    {
    \begin{adjustwidth}{1em}{0em}
    \noindent
    \textbf{#2} \hfill #1
    
    \vspace{0.1in}
    \noindent
    \ignorespaces
    }
    {
    \end{adjustwidth}
    }

\newenvironment{fact}[2][]
    {
    \index{#2}
    \hypertarget{#2}{\vspace{0.2in}}
    \begin{adjustwidth}{1em}{0em}
    \noindent
    \textbf{#2} \hfill #1
    
    \vspace{0.1in}
    \noindent
    \ignorespaces
    }
    {
    \end{adjustwidth}
    }

\title{Companion to Machine Learning}
\author{Rohan Kumar}
\date{}






\begin{document}

\maketitle
\newpage
\tableofcontents
\newpage

\section*{Introduction}

\subsection{What is Machine Learning}
    Machine Learning is the field of study that gives computers the ability to learn from data without being 
    explicitly programmed. This is good for problems that require a lot of fine-tuning or for which using a 
    traditional approach yields no good solution. Machine Learning's data dependency allows it to adapt to new data
    and gain insight for complex problems and large amounts of data.

\subsection{Applications of Machine Learning}
    Machine Learning can be used for a range of tasks and can be seen used in:
    \begin{itemize}
        \item Analyzing images of products on a productino line to automaticall classify them (Convolutional Neural Net)
        \item Forecasting company revenue based on performance metrics (Regression or Neural Net)
        \item Automatically classifying news articles (NLP using Recurrent Neural Networks)
        \item Summarizing long documents automatically (Natural Language Processing)
        \item Building intelligent bot for a game (Reinforcement Learning)
    \end{itemize}

\subsection{Types of Machine Learning}
    \begin{fact}[\homl]{Supervised Learning}
        In supervised learning, the training set you feed to the algorithm includes the desired solutions, called labels.
        (e.g determining if an email is spam would be trained a dataset of example emails labelled as spam or not spam.) \\[0.1in] 
        Some commonly used supervised learning algorithms are:
        \begin{itemize}
            \item k-Nearest Neighbors
            \item Linear Regression
            \item Logistic Regression
            \item Support Vector Machines (SVMs)
            \item Decision Trees and Random Forests
            \item Neural Networks
        \end{itemize}
    \end{fact}

    \begin{fact}[\homl]{Unsupervised Learning}
        In unsupervised learning, the training data is unlabeled and the system tries to learn without guidance. The system
        will try and automatically draw inferences and conclusions about the data and group it as such. (e.g. having a lot of
        data about blog visitors. Using a clustering algorithm we can group and detect similar visitors). \\[0.1in]
        Some important unsupervised learning algorithms are:
        \begin{itemize}
            \item Clustering
            \begin{itemize}
                \item K-Means
                \item DBSCAN
                \item Hierarchical Cluster Analysis
            \end{itemize}
            \item Anomaly detection and novelty detection
            \begin{itemize}
                \item One-class SVM
                \item Isolation Forest
            \end{itemize}
            \item Visualization and dimensionality reduction
            \begin{itemize}
                \item Principal Component Analysis (PCA)
                \item Kernel PCA
                \item Locally Linear Embedding (LLE)
                \item t-Distributed Stochastic Neighbor Embeedding (t-SNE)
            \end{itemize}
            \item Association rule learning
            \begin{itemize}
                \item Apriori
                \item Eclat
            \end{itemize}
        \end{itemize}
    \end{fact}

    \begin{fact}[\homl]{Semisupervised Learning}
        Labelling can be very time-consuming and costly, often there will be plenty of unlabelled and a few labelled instances.
        Algorithms that deal with data that is partially labeled is called semisupervised learning. A good example of this is
        Google Photos. Google clusters and groups your photos based on facial recognition (unsupervised) and then you can label one photo and it
        will be able to label every picture like that (supervised). Most semisupervised learning algorithms are combinations of unsupervised and 
        supervised algorithms.  
    \end{fact}

    \begin{fact}[\homl]{Reinforcement Learning}
        Reinforcement Learning is a learning algorithm based on a reward system. The learning system, called an agent, can observe the environment,
        select and perform actions, and get rewards in return (or penalites in the form of negative rewards). It will then learn by itself what the 
        best strategy, called a policy, to get the most reward over time. A policy defines what action the agent should choose when it is in a given
        situation.
    \end{fact}

\newpage

\section*{Sources}
    Throughout this compendium, each piece of information will be formatted as such.
    
    \vspace{0.1in}
    \begin{fact*}[Source]{Name / Description of fact}
        Information about fact.
    \end{fact*}
    \vspace{0.3in}
    
    \noindent The location which currently contains ``Source'' could potentially be filled with a variety of sources. Here is how to find the source based off the shortened form.
    
    \begin{itemize}
        \item \hypertarget{homl}{\textbf{Hands-On Machine Learning}} refers to Hands-On Machine Learning with Scikit-Learn, Keras, and TensorFlow, 2nd Edition by Aurélien Géron
    \end{itemize}

\newpage

\printindex

\end{document}
